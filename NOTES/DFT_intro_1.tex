%\documentclass[a4paper,11pt]{article} % print setting
\documentclass[a4paper,12pt]{article} % screen setting

\usepackage[a4paper]{geometry}
\geometry{verbose,tmargin=1.5cm,bmargin=1.5cm,lmargin=1.5cm,rmargin=1.5cm}

\setlength{\parskip}{\smallskipamount}
\setlength{\parindent}{0pt}

\usepackage{cmbright}
\renewcommand{\familydefault}{\sfdefault}

%\usepackage{mathptmx}
%\usepackage{sfmath}
%\usepackage{mathpazo}

\usepackage{fontspec}
%\setmonofont{FreeMono}

\usepackage{hyperref}
\usepackage{url}
\usepackage{xcolor}

\usepackage{amsmath}
\usepackage{amssymb}

\usepackage{graphicx}
\usepackage{float}

\usepackage{minted}
\newminted{julia}{breaklines,fontsize=\small}
\newminted{bash}{breaklines,fontsize=\small}
\newminted{text}{breaklines,fontsize=\small}

\newcommand{\txtinline}[1]{\mintinline{text}{#1}}
\newcommand{\jlinline}[1]{\mintinline{julia}{#1}}

\newmintedfile[juliafile]{julia}{breaklines,fontsize=\small}

\definecolor{mintedbg}{rgb}{0.90,0.90,0.90}
\usepackage{mdframed}

\BeforeBeginEnvironment{minted}{\begin{mdframed}[backgroundcolor=mintedbg]}
\AfterEndEnvironment{minted}{\end{mdframed}}

\usepackage{setspace}

\onehalfspacing

\usepackage{appendix}


\begin{document}


\title{Teori Fungsional Kerapatan}
\author{Fadjar Fathurrahman}
\date{}
\maketitle

\section{Interaksi Coulomb}

Energi interaksi antara dua elektron:
\begin{equation}
E_{\mathrm{ee}} = \frac{e^2}{4\pi\epsilon_{0}d_{ee}}
\end{equation}

Energi interaksi antara dua inti atom dengan nomor atom $Z$:
\begin{equation}
E_{\mathrm{nn}} = \frac{Z^2e^2}{4\pi\epsilon_{0}d_{\mathrm{nn}}}
\end{equation}

Energi interaksi antara elektron dan inti atom:
\begin{equation}
E_{\mathrm{en}} = -\frac{Ze^2}{4\pi\epsilon_{0}d_{\mathrm{en}}}
\end{equation}

Persamaan Schroedinger:
\begin{equation}
\left[ \frac{\mathbf{p}^2}{2m_{e}} + V(\mathbf{r}) \right] \psi(\mathbf{r}) = E\psi(\mathbf{r})
\end{equation}

Fungsi gelombang banyak-partikel:
\begin{equation}
\Psi =
\Psi\left(\mathbf{r}_{1}, \mathbf{r}_{2}, \ldots, \mathbf{r}_{N};
\mathbf{R}_{1}, \mathbf{R}_{2},\ldots,\mathbf{R}_{N_{\mathrm{at}}}
\right)
\end{equation}

Persamaan Schroedinger banyak-partikel:
\begin{equation}
\left[
-\sum_{i}\frac{\nabla^{2}_{i}}{2} -\sum_{i}\frac{\nabla^{2}_{I}}{2M_{I}}
-\sum_{i,I} \frac{Z_{I}}{\left| \mathbf{r}_{i} - \mathbf{R}_{I} \right|}
+\frac{1}{2}\sum_{i \neq j} \frac{1}{\left| \mathbf{r}_{i} - \mathbf{r}_{j} \right|}
+\frac{1}{2}\sum_{I \neq J} \frac{Z_{I} Z_{J}}{\left| \mathbf{R}_{i} - \mathbf{R}_{j} \right|}
\right] \Psi = E_{\mathrm{tot}} \Psi
\end{equation}

Aproksimasi clamped-nuclei:

\begin{equation}
E = E_{\mathrm{tot}} - \frac{1}{2} \sum_{I \neq J}
\frac{Z_{I} Z_{J}}{\left| \mathbf{R}_{I} - \mathbf{R}_{J} \right|}
\end{equation}

Persamaan untuk elektron:
\begin{equation}
\left[
-\sum_{i}\frac{\nabla^{2}_{i}}{2}
-\sum_{i,I} \frac{Z_{I}}{\left| \mathbf{r}_{i} - \mathbf{R}_{I} \right|}
+\frac{1}{2}\sum_{i \neq j} \frac{1}{\left| \mathbf{r}_{i} - \mathbf{r}_{j} \right|}
\right] \Psi = E \, \Psi
\end{equation}

Definisikan:
\begin{equation}
V_{\mathrm{n}} = -\sum_{I} \frac{Z_{I}}{\left| \mathbf{r} - \mathbf{R}_{I} \right|}
\end{equation}
sehingga
\begin{equation}
\left[
-\sum_{i}\frac{\nabla^{2}_{i}}{2}
+\sum_{i} V_{\mathrm{n}}(\mathbf{r}_{i})
+\frac{1}{2}\sum_{i \neq j} \frac{1}{\left| \mathbf{r}_{i} - \mathbf{r}_{j} \right|}
\right] \Psi = E \, \Psi
\end{equation}
Definisikan Hamiltonian banyak-elektron:
\begin{equation}
\hat{H}(\mathbf{r}_{1},\ldots,\mathbf{r}_{N}) =
\left[
-\sum_{i}\frac{\nabla^{2}_{i}}{2}
+\sum_{i} V_{\mathrm{n}}(\mathbf{r}_{i})
+\frac{1}{2}\sum_{i \neq j} \frac{1}{\left| \mathbf{r}_{i} - \mathbf{r}_{j} \right|}
\right]
\end{equation}
Persaman Schroedinger:
\begin{equation}
\hat{H} \, \Psi = E \, \Psi
\end{equation}

Hamiltonian elektron-tunggal:
\begin{equation}
\hat{H}_{0}(\mathbf{r}) = -\frac{1}{2}\nabla^2 + V_{n}(\mathbf{r})
\end{equation}

Sehingga Hamiltonian banyak-elektron juga dapat dituliskan menjadi:
\begin{equation}
\hat{H}(\mathbf{r}_{1},\ldots,\mathbf{r}_{N}) =
\sum_{i} \hat{H}_{0}(\mathbf{r}_{i}) +
\frac{1}{2}\sum_{i \neq j}\frac{1}{\left| \mathbf{r}_{i} - \mathbf{r}_{j} \right|}
\end{equation}


\section{Aproksimasi elektron independen}

Anggap interaksi Coulomb antara elektron tidak ada.

\begin{equation}
\sum_{i} \hat{H}_{0}(\mathbf{r}_{i})\, \Psi = E \, \Psi
\end{equation}

Aproksimasi:
\begin{equation}
\Psi(\mathbf{r}_{1},\mathbf{r}_{2},\ldots,\mathbf{r}_{N}) =
\phi_{1}(\mathbf{r}_{1}) \phi_{1}(\mathbf{r}_{2}) \ldots \phi_{N}(\mathbf{r}_{N})
\end{equation}

\begin{equation}
\hat{H}_{0}(\mathbf{r}_{i}) \phi_{i}(\mathbf{r}_{i}) = \varepsilon_{i}\phi_{i}
\end{equation}

\begin{equation}
\left[ \sum_{i} \hat{H}_{0}(\mathbf{r}_{i}) \right]
\phi_{1}(\mathbf{r}_{1}) \ldots \phi_{N}(\mathbf{r}_{N}) =
E\,\phi_{1}(\mathbf{r}_{1}) \ldots \phi_{N}(\mathbf{r}_{N})
\end{equation}

\begin{align*}
\left[ \hat{H}_{0}(\mathbf{r}_{1}) \phi_{1}(\mathbf{r})_{1} \right]
\phi_{2}(\mathbf{r}_{2}) \ldots \phi_{N}(\mathbf{r}_{N}) +
\phi_{1}(\mathbf{r}_{1}) \left[ \hat{H}_{0}(\mathbf{r}_{2}) \phi_{2}(\mathbf{r})_{2} \right]
\ldots \phi_{N}(\mathbf{r}_{N}) + \\
\phi_{1}(\mathbf{r}_{1}) \ldots \left[ \hat{H}_{0}(\mathbf{r}_{N}) \phi_{N}(\mathbf{r})_{N} \right]
= E\,\phi_{1}(\mathbf{r}_{1}) \ldots \phi_{N}(\mathbf{r}_{N})
\end{align*}
Sehingga:
\begin{equation}
E = \varepsilon_{1} + \varepsilon_{2} + \ldots + \varepsilon_{N}
\end{equation}

\section{Prinsip ekslusi Pauli}

\begin{equation}
\Psi(\mathbf{r}_{1},\mathbf{r}_{2}) = \frac{1}{\sqrt{2}}
\left[ \phi_{1}(\mathbf{r}_{1}) \phi_{2}(\mathbf{r}_{2}) -
\phi_{1}(\mathbf{r}_{2}) \phi_{2}(\mathbf{r}_{1})
\right]
\end{equation}

\section{Aproksimasi medan rata-rata}

Mean-field approximation




\bibliographystyle{unsrt}
\bibliography{BIBLIO}

\end{document}

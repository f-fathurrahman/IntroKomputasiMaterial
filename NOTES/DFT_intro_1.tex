%\documentclass[a4paper,11pt]{article} % print setting
\documentclass[a4paper,12pt]{article} % screen setting

\usepackage[a4paper]{geometry}
\geometry{verbose,tmargin=1.5cm,bmargin=1.5cm,lmargin=1.5cm,rmargin=1.5cm}

\setlength{\parskip}{\smallskipamount}
\setlength{\parindent}{0pt}

\usepackage{cmbright}
\renewcommand{\familydefault}{\sfdefault}

\usepackage{fontspec}
\setmonofont{FreeMono}

%\usepackage{sfmath}
%\usepackage{mathptmx}
%\usepackage{mathpazo}

\usepackage{hyperref}
\usepackage{url}
\usepackage{xcolor}

\usepackage{amsmath}
\usepackage{amssymb}

\usepackage{graphicx}
\usepackage{float}

\usepackage{minted}
\newminted{julia}{breaklines,fontsize=\small}
\newminted{bash}{breaklines,fontsize=\small}
\newminted{text}{breaklines,fontsize=\small}

\newcommand{\txtinline}[1]{\mintinline{text}{#1}}
\newcommand{\jlinline}[1]{\mintinline{julia}{#1}}

\newmintedfile[juliafile]{julia}{breaklines,fontsize=\small}

\definecolor{mintedbg}{rgb}{0.90,0.90,0.90}
\usepackage{mdframed}

\BeforeBeginEnvironment{minted}{\begin{mdframed}[backgroundcolor=mintedbg]}
\AfterEndEnvironment{minted}{\end{mdframed}}

\usepackage{setspace}

\onehalfspacing

\usepackage{appendix}


\begin{document}


\title{Teori Fungsional Kerapatan}
\author{Fadjar Fathurrahman}
\date{}
\maketitle

\section{Interaksi Coulomb}

Energi interaksi antara dua elektron:
\begin{equation}
E_{\mathrm{ee}} = \frac{e^2}{4\pi\epsilon_{0}d_{ee}}
\end{equation}

Energi interaksi antara dua inti atom dengan nomor atom $Z$:
\begin{equation}
E_{\mathrm{nn}} = \frac{Z^2e^2}{4\pi\epsilon_{0}d_{\mathrm{nn}}}
\end{equation}

Energi interaksi antara elektron dan inti atom:
\begin{equation}
E_{\mathrm{en}} = -\frac{Ze^2}{4\pi\epsilon_{0}d_{\mathrm{en}}}
\end{equation}

Persamaan Schroedinger:
\begin{equation}
\left[ \frac{\mathbf{p}^2}{2m_{e}} + V(\mathbf{r}) \right] \psi(\mathbf{r}) = E\psi(\mathbf{r})
\end{equation}

Fungsi gelombang banyak-partikel:
\begin{equation}
\Psi = 
\Psi\left(\mathbf{r}_{1}, \mathbf{r}_{2}, \ldots, \mathbf{r}_{N};
\mathbf{R}_{1}, \mathbf{R}_{2},\ldots,\mathbf{R}_{N_{\mathrm{at}}}
\right)
\end{equation}

\bibliographystyle{unsrt}
\bibliography{BIBLIO}

\end{document}
